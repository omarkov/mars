%%% File-Information {{{
%%% Filename: Mars_Latex_Vorlage.tex
%%% Purpose:  Die Vorlage f�r die Dokuemente des Mar-S StuPros
%%% Time-stamp: <2004-12-23 17:18:02 mp>
%%% Authors: RDE
%%% History:
%%% 2004-12-23 dokument erstellt
%%%
%%% Notes:
%%% - Bei �nderungen der Vorlage, m�ssen alle erstellten Dokumente entsprechend angepasst werden.
%%% - Es k�nnen die typischen deutschen Sonderzeichen dirket benutzt werden.
%%% - Anf�hrungszeichen bitte mit "` beginnen und mit "' abschlie�en.
%%% - chapters in eigene Dateien, nicht tiefer als subsubsection schachteln.
%%% - Texte unter chapter und section ist ok.
%%% - Kein Content in diese Datei
%%% - Erstellungsdatum, Dateiname, pdf-Name und Autor-Liste in dieser Datei f�llen.
%%% - �nderungshistorie erg�nzen 
%%%
%%%
%%% }}}
%%%
%%%%%%%%%%%%%%%%%%%%%%%%%%%%%%%%%%%%%%%%%%%%%%%%%%%%%%%%%%%%%%%%%%%%%%%%%%%%%%%%

\documentclass[	a4paper,%		% Papiergr��e A4
				12pt, %			
				BCOR10mm, %	
				DIV12, %		
				automark, %		% lebende Kolumnentitel
				normalheadings,%
				pagesize%			% Seitengr��e wird bei dvi und pdf richtig gesetzt
				]{scrreprt}		% entspricht article
%				]{scrartcl}		% entspricht article

\usepackage{ngerman, graphicx, color,verbatim}

%----------------  Schriften  ------------------------------------------------
%\usepackage[sc]{mathpazo} % Palatino mit echten small caps.
\usepackage[sc]{mathpazo} % Palatino mit echten small caps.
						  % Option osf f�r sc und old-style-figures

\usepackage[scaled=0.95]{helvet} % Sansserif-Schrift: Helvetica
\usepackage{courier}								% TeleType-Schrift Courier
\usepackage{lscape}
\usepackage{setspace}								% 1.05-facher Zeilenabstand wegen Palatino
\linespread{1.05}

\usepackage[T1]{fontenc}						% T1-Schriften verwenden
\usepackage[latin1]{inputenc}				% Zeichenkodierung latin 1
%-----------------------------------------------------------------------------

%---------------------------Titelseite--------------------------------------------
\titlehead{}
\subject{StuPro Mar-S}			% Typisierung: Unmittelbar �ber Titel
\title{\pdftitle} %\\ \vspace{0.3cm}\large v. \version}
\author{Mar-S Projektteam}
\date{\today}

%----------------------------Variablen f�llen---------------------------------------------------

\newcommand{\docname}{Spezifikation}					% Dateiname
\newcommand{\docdate}{\today}		% Erstellungsdatum
\newcommand{\pdftitle}{Spezifikation} 		%Name des Dokuments
\newcommand{\authorlist}{RD}  %Autorenliste
\newcommand{\version}{0.1}


%----------------  Hyperref  -------------------------------------------------
% Hyperref f�r navigierbare links und das setzen der Dokumenteigenschaften
\usepackage{hyperref}
\hypersetup{%
  pdftitle = { \pdftitle},
  pdfauthor = {\authorlist}
}

\setlength{\topmargin}{-0.7cm}
\setlength{\marginparwidth}{2.0 cm}
\setlength{\topskip}{0.7cm}
%\setlength{\footsep}{0.0cm}

%----------------  Kopf- und Fu�zeilen  --------------------------------------
\usepackage{scrpage2}
\pagestyle{scrheadings}
\renewcommand{\chapterpagestyle}{scrheadings}
%\ihead{\doctitle}\ohead{\docdate}
\ifoot{\docname}\ofoot{\docdate}
\setheadsepline{1pt}
\setfootsepline{1pt}
%-----------------------------------------------------------------------------

%\typearea[current]{calc} % Neuberechnung des Satzspiegels
%\reversemarginpar
%%%%%%%%%%%%%%%%%%%%%%%%%%%%%%%%%%%%%%%%%%%%%%%%%%%%%%%%%%%%%%%%%%%%%%%%%%%%%%
%%%%%%%%%%%%%%%%%%%%%%%%%%%%%%%%%%%%%%%%%%%%%%%%%%%%%%%%%%%%%%%%%%%%%%%%%%%%%%

%----------------------------Umgebungen / Commands----------------------------------------------------

%%%%%%%%%%%%%%%%%%%%%%%%%%%%%%%%%%%%%%%%%%%%%%%%%%%%%%%%%%%%%%%%%%%%%%%%%%%%%%
%%% Ein Use-Case-Environment hat 6 Parameter
%%% 1. Der Titel
%%% 2. Die Vorbedingungen
%%% 3. Der Normalablauf
%%% 4. Alternativabl�ufe
%%% 5. Nachbedingungen
%%%
\newenvironment{UseCase}[5]{ %Use-Case environment
	\subsection{#1}
	\label{#1}
	\minisec{Vorbedingung}
	#2
	\minisec{Normalablauf}
	#3
	\minisec{Alternativablauf}
	#4
	\minisec{Nachbedingung}
	#5
}
%%%
%%%%%%%%%%%%%%%%%%%%%%%%%%%%%%%%%%%%%%%%%%%%%%%%%%%%%%%%%%%%%%%%%%%%%%%%%%%%%%

%%%%%%%%%%%%%%%%%%%%%%%%%%%%%%%%%%%%%%%%%%%%%%%%%%%%%%%%%%%%%%%%%%%%%%%%%%%%%%
%%% Die �nderungshistorie ist immer in der Datei �nderungshistorie
%%% im gleichen Verzeichnis. Bitte dort die �nderungen pflegen.
%%%
\newcommand{\makehistorie}{
\input{�nderungshistorie}
\vspace{1cm}
\Large{\minisec{K�rzel}}
\large
\begin{tabbing}
XXXXXXXXXXXXXXXXXX\=XXXX\= \kill
Daniel Gerlach \> DG\\
Daniel Schleicher \>DS\\
Dominik R�ssler \>DR\\
Friedrich M�nsch \>FM\\
Oliver Markovic \>OM\\
Rajkumar Deshpande\>RD\\
Tim Rathgeber \> TR\\
Tobias Walter \>TW\\
Tomislav Jerkovic \>TJ\\
\end{tabbing}
}
%%%
%%%%%%%%%%%%%%%%%%%%%%%%%%%%%%%%%%%%%%%%%%%%%%%%%%%%%%%%%%%%%%%%%%%%%%%%%%%%%%
  
%----------------------------Einzelne Dateien zusammenbinden---------------------------------------------
\begin{document}

\maketitle
\makehistorie
\tableofcontents
\input{�berblick}
\input{UseCases}
\input{Rechtesystem}
\chapter{PDA}
\section{�berblick}
Der PDA wird dazu verwendet, die Grundfunktionen des Sinnesraumes zu steuern, wenn man sich in diesem befindet. Der PDA dient nicht dazu Profile zu ver�ndern.
\section{Funktionalit�ten}
Die Steuerung verf�gt �ber bestimmte Grundfunktionalit�ten. Diese Funktionalit�ten beziehen sich dabei sowohl auf die Steuerung der externen Komponenten als auch auf das MediaCenter (damit verbunden das Betriebssystem). Die Funktionalit�ten sind hierbei:
\begin{itemize}
\item Wechsel des eigenen Profils 
\item Logout aus dem System
\item Userwechsel (Logon)
\item Lautst�rkeregelung: kein Schieberegler; wie bei der Fernbedienung wird beim Bet�tigen des Lautst�rkebuttons sofort die Lautst�rke angepasst.
\item Auswahl des Lichtprogramms

\end{itemize}
\section{Umsetzung}
Die graphische Oberfl�che des PDAs wird als Weboberfl�che dargestellt. Diese Weboberfl�che sollte von den g�ngigsten Browsern (Opera, Microsoft IE, Firefox) unterst�tzt werden.\\
DIe Oberfl�che wird hierbei generisch gehalten. Sie bekommt also automatisch weitere Einstellungsm�glichkeiten, bei Erweiterung des Komplettsystems um weitere Komponenten.
\chapter{Datenbank}

\section{Verwendete Software}

\begin{itemize}

\item PostgreSQL Version 8.0 RC2 f�r Windows

\item Hibernate 3.0x
\end{itemize}

\section{PostgreSQL}

PostgreSQL ist die persistente SQL-Datenbank, in der die Benutzer des Systems und deren Profile gespeichert werden.

Ein Benutzer enth�lt:

\begin{itemize}

\item eindeutige MitarbeiterID als Prim�rschl�ssel
\item Abteilungsnummer
\item EMail-Adresse

\end{itemize}

Ein Profil enth�lt:

\begin{itemize}

\item eindeutige ProfilID als Prim�rschl�ssel
\item zugeh�rige MitarbeiterID
\item Alle Einstellungnen f�r die verf�gbaren Komponenten

\begin{itemize}
\item Licht: Farbe, Programm, Helligkeit
\item Musiklautst�rke
\item Playlist
\item Und weitere Komponenten, die hinzukommen k�nnen

\end{itemize}
\item ein Defaultflag

\end{itemize}

\section{Hibernate}

Um nicht von SQL-Befehlen abh�ngig zu sein, wird Hibernate als Object/Relation-Mapper benutzt.
Es werden Java-Methoden zur Verf�gung gestellt, die SQL Befehle ersetzen. Somit ist eine performantere Implentierung gew�hleistet.
\input{Begriffslexikon}

%Hier die einzelnen Dateien angeben. Nur Name, ohne Endung. BSP:
%\input{Chapter1}


\end{document}

%--------------------------------------------------------------------------------------------------------

