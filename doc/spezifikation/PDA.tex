\chapter{PDA}
\section{�berblick}
Der PDA wird dazu verwendet, die Grundfunktionen des Sinnesraumes zu steuern, wenn man sich in diesem befindet. Der PDA dient nicht dazu Profile zu ver�ndern.
\section{Funktionalit�ten}
Die Steuerung verf�gt �ber bestimmte Grundfunktionalit�ten. Diese Funktionalit�ten beziehen sich dabei sowohl auf die Steuerung der externen Komponenten als auch auf das MediaCenter (damit verbunden das Betriebssystem). Die Funktionalit�ten sind hierbei:
\begin{itemize}
\item Wechsel des eigenen Profils 
\item Logout aus dem System
\item Userwechsel (Logon)
\item Lautst�rkeregelung: kein Schieberegler; wie bei der Fernbedienung wird beim Bet�tigen des Lautst�rkebuttons sofort die Lautst�rke angepasst.
\item Auswahl des Lichtprogramms

\end{itemize}
\section{Umsetzung}
Die graphische Oberfl�che des PDAs wird als Weboberfl�che dargestellt. Diese Weboberfl�che sollte von den g�ngigsten Browsern (Opera, Microsoft IE, Firefox) unterst�tzt werden.\\
DIe Oberfl�che wird hierbei generisch gehalten. Sie bekommt also automatisch weitere Einstellungsm�glichkeiten, bei Erweiterung des Komplettsystems um weitere Komponenten.