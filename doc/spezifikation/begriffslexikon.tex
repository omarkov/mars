\chapter{Begriffslexikon}

\begin{tabular}{p{3 cm}p{10cm}}

Administrator	\label{Administrator} 
		&	Ein Benutzer der alle Aktionen im System ausf�hren darf. Er hat Zugriff auf alle Benutzerdaten. Der Administrator verwaltet das System.\\
				
Anmeldung \label {Anmeldung}
		& Eine Anmeldung besteht aus Benutzername und Benutzerpasswort. Mit diesen Daten kann sich ein Benutzer am System anmelden.\\

Atomare Daten/Dateien \label{Atomare Daten/Dateien} 
		&	Multimediadateien, wie mp3, wma, jpg, avi, mpg, usw. \\

Benutzer \label{Benutzer} 
		&	Ein Benutzer ist eine Person die das System bedient, ein Benutzer hat dabei genau eine Benutzerrolle (Administrator, Standard-User, Gast) Oft ist damit aber auch ein logischer Benutzer gemeint. Ein logischer Benutzer ist das Model eines reellen Benutzers\\
		
Benutzerdaten \label{Benutzerdaten} 
		&	Spezifische Daten eines Benutzers wie Anmeldung, Name, Vorname, Benutzerprofil \\
		
Benutzername \label{Benutzername} 
		&	Ein Name f�r einen Benutzer der diesen eindeutig identifiziert. Nicht der Name des Benutzers.\\

Benutzer-passwort \label{Benutzerpasswort} 
		&	Eine Zeichenkombination.\\

Benutzerprofil \label{Benutzerprofil} 
		&	Paket aus Benutzerprofilelementen. Jedes Benutzerprofil hat 0,1 Playlisten.\\
		
Benutzer-profilelement \label{Benutzerprofilelement} 
		&	Das sind Licht- und Lautst�rkeeinstellungen und Playlists. Dies soll erweiterbar sein.\\
		
Daten \label{Daten} 
		& Je nach Kontext u.a. Oberbegriff f�r Playlists, Raumprofile, atomare Dateien, Datenstreams.\\

Gast \label{Gast} 
		&	Ein Gast ist ein Standard-User, der �ber keine Quota verf�gt.  \\		
		
Gruppe \label{Gruppe} 
		&	Zusammenfassung von Benutzern, die gemeinsame Sicht auf Daten haben.\\
		
Gruppendaten \label{Gruppendaten}
		&	Spezifische Daten der Gruppe wie Name der Gruppe oder Benutzer die zu der Gruppe geh�ren.\\
		
Gruppenprofil \label{Gruppenprofil} 
		&	Ein Gruppenprofil ist ein Benutzerprofil das allen Mitgliedern einer Gruppe zug�nglich ist.\\
		
Gruppenprofil-element \label{Gruppenprofilelement} 
		&	Diese Elemente sind wie Benutzerprofilelemente, aber sie sind jedem Mitglied einer Gruppe zug�nglich.\\
\end{tabular}

\begin{tabular}{p{3 cm}p{10cm}}

Pflichtfeld  \label{Pflichtfeld} 
		&	Ein Pflichtfeld ist ein Feld das auf jeden Fall gef�llt werden muss. Bsp. Benutzername einer Benutzers.\\

Playlist \label{Playlist} 
		&	Zusammenstellung von Referenzen auf atomare Dateien.\\

Profil \label{Profil} 
		&	Zusammenfassung f�r Playlist und Raumprofil. Ein Profil besteht dabei aus genau einer Playlist und einem Raumprofil)\\

Profilelement \label{Profilelement} 
		&	Ein Profilelement ist ...\textcolor{red}{TBD}TBD \\
	
Raumprofil \label{Raumprofil} 
		&	Zusammenstellung von Einstellungsparametern, die sich auf steuerbare Komponenten des Smartrooms beziehen.)\\

Rechte \label{Rechte} 
		&	Folgende Rechte werden implementiert \textbf{R} = Read ; \textbf{W} =Write (beinhaltet Modify, Create, Delete); \textbf{A}= Administrate\\

Sicht \label{Sicht} 
		&	Oberbegriff f�r eine Sichtbarkeitsklasse auf Daten (Public, Group, Private) \\	
		
Smartroomprofil \label{Smartroomprofil} 
		&	Siehe Profil.\\

Standard-User \label{Standard-User} 
		&	Benutzer der alle nichtadministrativen Funktionen steuern kann. \\

%x \label{x} 
%		&	x \\		
\end{tabular}
