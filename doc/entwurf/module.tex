%%% -*- Mode: LaTeX; indent-tabs-mode: nil -*-
%%%
%%% Copyright (c) 2004, Oliver Markovic <entrox@entrox.org> 
%%%   All rights reserved. 
%%%
%%% Redistribution and use in source and binary forms, with or without
%%% modification, are permitted provided that the following conditions are met:
%%%
%%%  o Redistributions of source code must retain the above copyright notice,
%%%    this list of conditions and the following disclaimer. 
%%%  o Redistributions in binary form must reproduce the above copyright
%%%    notice, this list of conditions and the following disclaimer in the
%%%    documentation and/or other materials provided with the distribution. 
%%%  o Neither the name of the author nor the names of the contributors may be
%%%    used to endorse or promote products derived from this software without
%%%    specific prior written permission. 
%%%
%%% THIS SOFTWARE IS PROVIDED BY THE COPYRIGHT HOLDERS AND CONTRIBUTORS "AS IS"
%%% AND ANY EXPRESS OR IMPLIED WARRANTIES, INCLUDING, BUT NOT LIMITED TO, THE
%%% IMPLIED WARRANTIES OF MERCHANTABILITY AND FITNESS FOR A PARTICULAR PURPOSE
%%% ARE DISCLAIMED.  IN NO EVENT SHALL THE COPYRIGHT OWNER OR CONTRIBUTORS BE
%%% LIABLE FOR ANY DIRECT, INDIRECT, INCIDENTAL, SPECIAL, EXEMPLARY, OR
%%% CONSEQUENTIAL DAMAGES (INCLUDING, BUT NOT LIMITED TO, PROCUREMENT OF
%%% SUBSTITUTE GOODS OR SERVICES; LOSS OF USE, DATA, OR PROFITS; OR BUSINESS
%%% INTERRUPTION) HOWEVER CAUSED AND ON ANY THEORY OF LIABILITY, WHETHER IN
%%% CONTRACT, STRICT LIABILITY, OR TORT (INCLUDING NEGLIGENCE OR OTHERWISE)
%%% ARISING IN ANY WAY OUT OF THE USE OF THIS SOFTWARE, EVEN IF ADVISED OF THE
%%% POSSIBILITY OF SUCH DAMAGE.

\chapter{Externe Module}

\section{�bersicht}

Es soll die M�glichkeit bestehen, externe Module an das laufende Mar-S
System ohne �nderungen an der zentralen Komponente einzuf�gen. Um eine
m�glichst grosse Flexibilit�t bez�glich verwendeter Sprache und physikalischer
Lage zu gew�hrleisten, kommunizieren diese Module �ber ein Klartextprotokoll
mit der Zentralkomponente (definiert in Kapitel TK).
\\\\
Externe Module sind das Bindeglied zwischen existierenden Ger�ten mit den
dazugeh�rigen propriet�ren Protokollen (Beleuchtung, OPS), anderen Programmen
(Media Center) und dem Rest des Mar-S Systems. Zu diesem Zwecke k�nnen
Parameter und Kommandos �ber ein Netzwerk gesetzt werden.


\section{Parameter}

Einzelne Module exportieren Parameter, die von der Zentralkomponente gesetzt
und auch ausgelesen werden k�nnen. Eine Beschreibung aller m�glichen Parameter
kann jederzeit vom Controller angefordert werden.
\newline
Parameter werden durch eine Kennung und durch eine Liste von Attributen
charakterisiert, wobei Name, Voreinstellung und Flags bei allen Parametertypen
vorhanden sind. Flags bestehen aus einer Zeichenkette, in der gespeichert ist,
ob man diesen Parameter lesen und setzen (``rw''), nur lesen (``r'') oder nur
setzen (``w'') kann.


\subsection{Boole'sch}

Ein einfacher Boole'scher Parameter mit zwei m�glichen Zust�nden: An und Aus.
\newline

\begin{tabular}{c|c|l}
Attribut & Typ & Beschreibung\\
\hline
Name & string & Der Name dieses Parameters\\
Default & bool & Die Voreinstellung dieses Parameters\\
Flags & string & Flags\\
\end{tabular}


\subsection{Numerisch}

Numerische Parameter stellen einen Zahlenwert in einem bestimmten Wertebereich
mit einer separaten Schrittweite dar.
\newline

\begin{tabular}{c|c|l}
Attribut & Typ & Beschreibung\\
\hline
Name & string & Der Name dieses Parameters\\
Default & int & Die Voreinstellung dieses Parameters\\
Flags & string & Flags\\
Min & int & Untere Schranke des Wertebereichs\\
Max & int & Obere Schranke des Wertebereichs\\
Step & int & Schrittweite\\
\end{tabular}


\subsection{Zeichenkette}

Eine einfache Zeichenkette von beliebiger L�nge.
\newline

\begin{tabular}{c|c|l}
Attribut & Typ & Beschreibung\\
\hline
Name & string & Der Name dieses Parameters\\
Default & string & Die Voreinstellung dieses Parameters\\
Flags & string & Flags\\
\end{tabular}


\subsection{Liste}

Eine Liste aus Zeichenketten.
\newline

\begin{tabular}{c|c|l}
Attribut & Typ & Beschreibung\\
\hline
Name & string & Der Name dieses Parameters\\
Default & string* & Die Voreinstellung dieses Parameters\\
Flags & string & Flags\\
\end{tabular}


\section{Kommandos}

Kommandos bzw. Nachrichten dienen dazu, bestimmte Aktionen auszuf�hren, wobei
das sowohl auf Seite eines externen Moduls als auch auf der Zentralkomponente
sein kann. Beispiele daf�r sind ``Tag 123 neu dazugekommen'',
``Starte Wiedergabe der aktuellen Playliste'' oder ``User ausloggen''. Ein
Kommando kann beliebig viele Parameter und einen R�ckgabewert als Zeichenkette
haben. Die genaue Kodierung ist in Kapitel TK beschrieben.\\
Jedes externe Modul muss das Standardkommando ``Beschreibung'' unterst�tzen,
welches die exportierte Schnittstelle an die Zentrale schickt.


\section{Beschreibung eines Moduls}

Externe Module k�nnen aufgefordert werden, ihre exportierte Schnittstelle
(Parameter und Kommandos) an die zentrale Komponente weiterzugeben. Das genaue
Format dieser Beschreibung ist in Kapitel TK definiert. Inhaltlich umfasst die
Beschreibung Informationen "uber das gesamte Modul, jeden einzelnen Parameter
mit den dazugeh"origen Attributen und jedes Kommando mit der jeweiligen Signatur.
