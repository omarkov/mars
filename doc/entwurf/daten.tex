%%% -*- Mode: LaTeX; indent-tabs-mode: nil -*-
%%%
%%% Copyright (c) 2004, Oliver Markovic <entrox@entrox.org> 
%%%   All rights reserved. 
%%%
%%% Redistribution and use in source and binary forms, with or without
%%% modification, are permitted provided that the following conditions are met:
%%%
%%%  o Redistributions of source code must retain the above copyright notice,
%%%    this list of conditions and the following disclaimer. 
%%%  o Redistributions in binary form must reproduce the above copyright
%%%    notice, this list of conditions and the following disclaimer in the
%%%    documentation and/or other materials provided with the distribution. 
%%%  o Neither the name of the author nor the names of the contributors may be
%%%    used to endorse or promote products derived from this software without
%%%    specific prior written permission. 
%%%
%%% THIS SOFTWARE IS PROVIDED BY THE COPYRIGHT HOLDERS AND CONTRIBUTORS "AS IS"
%%% AND ANY EXPRESS OR IMPLIED WARRANTIES, INCLUDING, BUT NOT LIMITED TO, THE
%%% IMPLIED WARRANTIES OF MERCHANTABILITY AND FITNESS FOR A PARTICULAR PURPOSE
%%% ARE DISCLAIMED.  IN NO EVENT SHALL THE COPYRIGHT OWNER OR CONTRIBUTORS BE
%%% LIABLE FOR ANY DIRECT, INDIRECT, INCIDENTAL, SPECIAL, EXEMPLARY, OR
%%% CONSEQUENTIAL DAMAGES (INCLUDING, BUT NOT LIMITED TO, PROCUREMENT OF
%%% SUBSTITUTE GOODS OR SERVICES; LOSS OF USE, DATA, OR PROFITS; OR BUSINESS
%%% INTERRUPTION) HOWEVER CAUSED AND ON ANY THEORY OF LIABILITY, WHETHER IN
%%% CONTRACT, STRICT LIABILITY, OR TORT (INCLUDING NEGLIGENCE OR OTHERWISE)
%%% ARISING IN ANY WAY OUT OF THE USE OF THIS SOFTWARE, EVEN IF ADVISED OF THE
%%% POSSIBILITY OF SUCH DAMAGE.

\chapter{Datenschicht}
\section{Allgemein}
Im folgenden wir das Datenmodell des Mar-s Systems beschrieben. Die Beschreibung orientiert sich am UML-Klassendiagramm des zu entwerfenden Systems. Das Datenbankschema wird nicht weiter beschrieben, da es aus dem Datenmodell (Klassendiagramm) generiert wird. Das Mapping zwischen den Datenklassen des Klassendiagramms und den Datenbanktabellen �bernimmt das Mappingframework \textbf{HIBERNATE}. 
\newpage

\section{Datenmodell/Klassendiagramm}
Aus Gr�nden der �bersichtlichkeit sind die Methoden der Klassen im nachfolgenden Diagramm ausgeblendet. 
\begin{figure}[!h]
\begin{center}
  \includegraphics[width=360pt]{Klassendiagramm.png}
  \caption{Klassendiagramm}
\end{center}
\vspace*{-6mm}
\end{figure}


\section{Beschreibung der Datenklassen}

\subsection{Klasse: \textit{Department}}
Enth�lt Informationen zu den Abteilungen in der Einsatzumgebung des Mar-s Systems. Dient der Verwaltung der User. Verwaltet werden Abt.-name, -k�rzel und ein Kommentarfeld. 

\subsection{Klasse: \textit{User}}
Speichert alle Benutzers die im Mar-S System registriert sind. 
Au�erdem beinhaltet sie alle wichtigen Daten um die Anmeldung am Betriebssystem (MS Windows XP Mediacenter Edition), 
die Zuordnung zwischen Identifications (Anmeldungen an den SmartRoom) und Mar-s-User sowie dem Autostart von Profilen zu realisieren. Zus�tzlich ist die Quota des Users in dieser Klasse hinterlegt. Das Passwort wird aus Sicherheitsgr�nden in verschl�sselter Form in der Datenbank gespeichert (Symmetrisches Verschl�sselungsverfahren), eine Entschl�sselung muss Betriebssystembedingt (vgl. Windows-API Systemlogin) gew�hrleistet werden. �ber die Klasse \textit{User} ist der direkte Zugriff auf SmartRoomProfile m�glich.
  

\subsection{Klasse: \textit{Group}}
Diese Klassen beschreibt die Gruppen mit einem eindeutigen Name und einem Kommentarfeld. Die Gruppen verf�gen �ber ein Haltbarkeitsdatum. Wird es �berschritten wird die Gruppe sowie aller Referenzen zu ihr gel�scht.
Benutzer des Mar-s Systems k�nnen SmartRoomProfile definieren die f�r alle Mitlieder einer bestimmten Gruppe sichtbar sind. Jedes Objekt der Klasse \textit{UserLogin} geh�rt mindestens der Gruppe \underline{\textit{Public}} an.

\subsection{Klasse: \textit{LogInSystem}}
Diese Klasse repr�sentiert LogInSysteme mit denen sich ein Benutzer an den SmartRoom anmelden kann. Ein LogInSystem ist das OPS ein anderes der PDA

\subsection{Klasse: \textit{Identification}}
Identifications bestehen aus einem LogInSystem und einem f�r das LogInSystem eindeutige tagID. Jeder Benutzer kann mehrere Identifications haben.



\subsection{Klasse: \textit{SmartRoomComponent}}
Die Klasse repr�sentiert eine Komponente im SmartRoom. Eine Komponente kann mehrere Attribute (ComponenentAttributes) besitzen.

\subsection{Klasse: \textit{ComponentAttribute}}
Die Klasse beschreibt die Attribute einer Komponente bzw. die Subattribute eines �bergeordneten Attributs.
Jedes Komponentenattribut besitzt einen Wert (Siehe klasse \textit{ComponentAttributeValue}). Die �nderbarkeit der Attributwerte kann �ber das Feld editable eingeschr�nkt werden, dies ist wichtig f�r die Generierung der JSP-Seiten um Komponentenattributwerte zu modifizieren (Z.B. Lichtintensit�t) oder nur darzustellen (z.B. maximale Lichtintensit�t). Jedes Attribut hat einen bestimmten Typ (AttributeType).

\subsection{Klasse: \textit{AttributeType}}
Die Subklassen von AttributeType legen die Typen eines Attributs fest. Bisher kann ein Attribut mit Werten vom Typ String, Numeric, Boolean und List bef�llt sein. Im Attributtyp ist auch der Default-Wert f�r das Attribut fest gelegt.

\subsection{Klasse: \textit{SmartRoomProfil}}
Diese Klasse enth�lt die Raumprofile f�r den Smartroom, diese beinhalten alle Komponenten des Smartrooms. Diese Klasse erm�glicht den Zugriff auf die einzelnen Komponenten. Ein Profil besteht also aus lauter Einstellungen von Componenteneigenschaften. Diese werden in ComponentSettings zusammengefasst.

\subsection{Klasse: \textit{ComponentSetting}}
ComponentSettings fassen die konkreten Einstellungen eines Profils (SmartRoomProfile) f�r eine Komponente (SmartRoomComponent) zusammen.


\subsection{Klasse: \textit{ComponentAttributeValue}}
Instanzen dieser Klasse realisieren die Werte die das verkn�pfte ComponentAttribute annehmen soll. Wie die AttributTypen k�nnen die Values auch vom Typ Numeric, String, Boolean und List sein.
