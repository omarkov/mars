%%% -*- Mode: LaTeX; indent-tabs-mode: nil -*-
%%%
%%% Copyright (c) 2004, Oliver Markovic <entrox@entrox.org> 
%%%   All rights reserved. 
%%%
%%% Redistribution and use in source and binary forms, with or without
%%% modification, are permitted provided that the following conditions are met:
%%%
%%%  o Redistributions of source code must retain the above copyright notice,
%%%    this list of conditions and the following disclaimer. 
%%%  o Redistributions in binary form must reproduce the above copyright
%%%    notice, this list of conditions and the following disclaimer in the
%%%    documentation and/or other materials provided with the distribution. 
%%%  o Neither the name of the author nor the names of the contributors may be
%%%    used to endorse or promote products derived from this software without
%%%    specific prior written permission. 
%%%
%%% THIS SOFTWARE IS PROVIDED BY THE COPYRIGHT HOLDERS AND CONTRIBUTORS "AS IS"
%%% AND ANY EXPRESS OR IMPLIED WARRANTIES, INCLUDING, BUT NOT LIMITED TO, THE
%%% IMPLIED WARRANTIES OF MERCHANTABILITY AND FITNESS FOR A PARTICULAR PURPOSE
%%% ARE DISCLAIMED.  IN NO EVENT SHALL THE COPYRIGHT OWNER OR CONTRIBUTORS BE
%%% LIABLE FOR ANY DIRECT, INDIRECT, INCIDENTAL, SPECIAL, EXEMPLARY, OR
%%% CONSEQUENTIAL DAMAGES (INCLUDING, BUT NOT LIMITED TO, PROCUREMENT OF
%%% SUBSTITUTE GOODS OR SERVICES; LOSS OF USE, DATA, OR PROFITS; OR BUSINESS
%%% INTERRUPTION) HOWEVER CAUSED AND ON ANY THEORY OF LIABILITY, WHETHER IN
%%% CONTRACT, STRICT LIABILITY, OR TORT (INCLUDING NEGLIGENCE OR OTHERWISE)
%%% ARISING IN ANY WAY OUT OF THE USE OF THIS SOFTWARE, EVEN IF ADVISED OF THE
%%% POSSIBILITY OF SUCH DAMAGE.

\chapter{Kommunikationsprotokoll}

\section{Verbindungsaufbau / Service Discovery}
Das Auffinden der verf"ugbaren Dienste erfolgt "uber den Internetstandard
``ZeroConf''. Hierbei bietet die zentrale Komponente einen ``Mar-S-Master'',
die externen Module einen  ``Mar-S-Slave'' - Dienst an. Nach dem erfolgreichen 
Finden eines Masters bzw. Slaves erfolgt der Verbindungsaufbau. Die Verbindungen 
bleiben bis zum abschalten einer Komponente offen (persistente Sockets). Der
Auffindungsprozess kann jeder Zeit erfolgen. 
\section{Allgemeiner Kommunikationsablauf}
Die Kommunikation erfolgt "uber Nachrichten. Es gibt zwei Arten von Nachrichten:
einzeilige und mehrzeilige Nachrichten. Eine einzeilige Nachricht wird durch ein Linefeet 
terminiert. Eine mehrzeile Nachricht wird durch ein
BEGIN <NACHRICHTEN-TAG> eingeleitet und durch ein END <NACHRICHTEN-TAG> abgeschlossen.
Jede Nachricht wird entweder durch ein OK <Antwort> bei Erfolg, oder durch ein ERROR <Fehlermedung>
quittiert. 
\section{Registrierung}
Nach erfolgreichem Verbindungsaufbau sendet der Slave seine Protokollversion,
eine systemweit eindeutige Modulekennung und die Serienummer seiner Schnittstellenbeschreibung.

\begin{center}
\begin{tabular*}{350pt}{l}
HELO <Protokollversion> <Modulkennung> <Seriennummer>\\
\end{tabular*}
\end{center}

Fehler k"onnen durch etweder eine inkompatible Protokollversion oder durch eine nichteindeutige 
Servicekennung eintreten. 

Mit dem Befehl MODULELIST kann ein Liste aller registrierter Slaves abgefragt werden.

\begin{center}
\begin{tabular*}{350pt}{l}
S: MODULELIST \\
M: BEGIN MODULELIST \\
M: MODULE <Modulkennung>\\
M: ...\\
M: END MODULELIST\\
\end{tabular*}
\end{center}

Nach einem erfolgreichen ``Handshake'' fordert der Master bei nicht mehr "ubereinstimmender 
Seriennummer die Schnittstellenbeschreibung beim Slave an. Ebenso kann ein Slave die Beschreibungsliste
eines anderen Slaves beim Master anfordern.

\begin{center}
\begin{tabular*}{350pt}{l}
DESCRIBE [<Modulkennung>]\\ 
\end{tabular*}
\end{center}

Der Slave sendet danach asynchron sein Beschreibung. Die Beschreibungselemente k"onnen 
in beliebiger Reihenfolge und Zusammensetzung verwendet werden.

\begin{center}
\begin{tabular*}{350pt}{l}
BEGIN DESCRIPTION \\
NAME <Name des Modules>\\
BOOL <Name des Parameter> <Default> <Flags>\\
NUMERIC <Name des Parameter> <Default> <Flags> <Min> <Max> <Step>\\
STRING <Name des Parameter> <Default> <Flags>\\
LIST <Name des Parameter> <DefaultIdx> <Flags>\\
BEGIN COMMAND <Name des Commands> <ReturnType>\\
PARAMETER <Type> <Name>\\
END COMMAND\\
END DESCRIPTION\\
\end{tabular*}
\end{center}

\section{Parameter setzen/lesen}

Nach abeschlossener Registrierung kann der Master exportierte Parameter setzen und auslesen.
Bei unpassender Ausf"uhrung (z.B. schreiben auf RO-Wert) oder unbekkanntem Parameter wird 
ein Fehler zur"uckgeliefert. 

\begin{center}
\begin{tabular*}{350pt}{l}
M: GET [<Modulkennung>] <Name des Parameters>\\
C: OK \\ 
C: <Wert>\\
bzw. \\
C: BEGIN LIST\\
C: LISTELEMENT <Value>\\
...\\
C: END LIST\\
\\
SET <Name des Parameters> <Wert>\\
\end{tabular*}
\end{center}

\section{Kommando aufrufen}

Kommandos k"onnen von beiden Komponenten gesendet werden. Bei ung"ultiger Paramterliste bzw. 
unbekannten Kommando wird ein Fehler zur"uckgeliefert.

\begin{center}
\begin{tabular*}{350pt}{l}
BEGIN CALL [<Modulkennung>] <Name des Kommandos>\\
PARAMETER <Name> <value>\\
END CALL\\
\end{tabular*}
\end{center}

\section{Verbindungsabbau}

Verbindungen k"onnen von beiden Komponenten entweder explizit mit BYE, bzw. implizit durch 
schlie"sen der Socket beendet werden, wobei ein explizter Verbindungsaubau in jedem Fall
vorzuziehen ist.
