%%% -*- Mode: LaTeX; indent-tabs-mode: nil -*-
%%%
%%% Copyright (c) 2004, Oliver Markovic <entrox@entrox.org> 
%%%   All rights reserved. 
%%%
%%% Redistribution and use in source and binary forms, with or without
%%% modification, are permitted provided that the following conditions are met:
%%%
%%%  o Redistributions of source code must retain the above copyright notice,
%%%    this list of conditions and the following disclaimer. 
%%%  o Redistributions in binary form must reproduce the above copyright
%%%    notice, this list of conditions and the following disclaimer in the
%%%    documentation and/or other materials provided with the distribution. 
%%%  o Neither the name of the author nor the names of the contributors may be
%%%    used to endorse or promote products derived from this software without
%%%    specific prior written permission. 
%%%
%%% THIS SOFTWARE IS PROVIDED BY THE COPYRIGHT HOLDERS AND CONTRIBUTORS "AS IS"
%%% AND ANY EXPRESS OR IMPLIED WARRANTIES, INCLUDING, BUT NOT LIMITED TO, THE
%%% IMPLIED WARRANTIES OF MERCHANTABILITY AND FITNESS FOR A PARTICULAR PURPOSE
%%% ARE DISCLAIMED.  IN NO EVENT SHALL THE COPYRIGHT OWNER OR CONTRIBUTORS BE
%%% LIABLE FOR ANY DIRECT, INDIRECT, INCIDENTAL, SPECIAL, EXEMPLARY, OR
%%% CONSEQUENTIAL DAMAGES (INCLUDING, BUT NOT LIMITED TO, PROCUREMENT OF
%%% SUBSTITUTE GOODS OR SERVICES; LOSS OF USE, DATA, OR PROFITS; OR BUSINESS
%%% INTERRUPTION) HOWEVER CAUSED AND ON ANY THEORY OF LIABILITY, WHETHER IN
%%% CONTRACT, STRICT LIABILITY, OR TORT (INCLUDING NEGLIGENCE OR OTHERWISE)
%%% ARISING IN ANY WAY OUT OF THE USE OF THIS SOFTWARE, EVEN IF ADVISED OF THE
%%% POSSIBILITY OF SUCH DAMAGE.

\chapter{OPS}
\section{Einleitung}
Im Sinnesraum gibt es generell zwei M�glichkeiten sich am System anzumelden. 
Die eine Variante ist die tagbasierte, die andere die PDA-basierte.
Bei beiden Varianten hat der Benutzer das System zu steuern. Wie der Login-Prozess 
genau auf Betriebssystemebene abl�uft, ist im Kapitel Betriebssystem nachzulesen.\\
Um ein Benutzer tagbasiert einloggen zu k�nnen, mu� das System wissen, wer sich im Raum befindet.
\section{Tag-Logik}
Die einzelnen Benutzer verf�gen �ber Tags, welche alle 1,5 Sekunden ein Signal an die Empfangsantenne schicken.
\\
Wird nun ein Benutzer zweimal innerhalb von 6 Sekunden erkannt so wird er als im Raum befindlich gesehen.\\
Ist dies nicht der Fall, so wird er nicht als im Raum befindlich gesehen. Die Komponente OPS schickt also alle sechs Sekunden eine Nachricht an den Controller, ob eine Person den Raum verlassen oder betreten hat.
\section{Klassen}
Die Komponente OPS besteht quasi aus 3 Klassen.
\begin{itemize}
\item Eine Klasse stellt eine Verbindung zu einer seriellen Schnittstelle her.
\item Eine Klasse erh�lt die Nachrichten von der seriellen Schnitstelle.
\item Eine Klasse enth�lt die Logik, ob sich ein Benutzer im Raum befindet oder nicht.
\end{itemize}




