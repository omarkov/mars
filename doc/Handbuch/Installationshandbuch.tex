\documentclass[12pt,oneside,german,a4paper]{article}
\usepackage[latin1]{inputenc}
%\usepackage{isolatin1}
\usepackage{a4}
\usepackage[T1]{fontenc}
\usepackage{babel}
\usepackage{german}
\usepackage{graphicx}
\usepackage{newcent}        % Umstellung auf Schrift Times, Wichtig f�r gute
                           % PDF Qualit�t.


\usepackage{makeidx}
\makeindex 
%----------------
% Definitionen

\setlength{\textwidth}{16cm} \setlength{\textheight}{23cm}
\setlength{\oddsidemargin}{0cm} \setlength{\topmargin}{0cm}
\setlength{\parindent}{0cm} \setlength{\parskip}{1.5ex}


\begin{document}
\onecolumn
%---------------------------------------
% Titelseite & Inhaltsverzeichnis

\title{\textbf{Installationshandbuch}\\
      \textbf{Mar-S}}
%\author{Projektteam des Studienprojekts ''Mar-S''}
\date{\today}
% heutiges Datum: \date{\today}
\maketitle
\begin{center}
Fraunhofer Institut Office Innovation Center\\
Universit�t Stuttgart\\
Rosensteinstr. 22-24\\
70191 Stuttgart\\
\end{center}

\newpage


%--------------------------------------------------------------------------------------------------
% Inhaltsverzeichnis

\tableofcontents
%--------------------------
% Hier kommt der Inhalt
%--------------------------------------------------------------------------------------------------

\newpage

\section{Systemvoraussetzungen}

\subsection{Medienrechner}

Es wird ben�tigt:

\begin{enumerate}

\item Vorinstalliertes und lauff�higes Linux-System
\item Java 1.5 Umgebung
\item Samba 3.x
\item Freevo
\begin{enumerate}
\item Mplayer
\item Python 2.3 oder h�her
\end{enumerate}

\item PostgreSQL 7.4
\end{enumerate}

\subsection{Webrechner}

Medienrechner und Webrechner k�nnen der gleiche Rechner sein. Auf diesem muss TomCat 5.x laufen.

\newpage

\section{Inbetriebnahme}

Den Inhalt der gelieferten CD in das gew�nschte Verzeichnis(<mars\_install>) kopieren (muss mit Rootrechten geschehen).

\subsection{Starten und Stoppen des Systems}

F�r den korrekten Ablauf muss alles konfiguriert sein (siehe Abschnitt~\ref{conf}).
Es wird ein Skript geliefert, das im Installationsordner ausgef�hrt werden muss (muss mit Rootrechten geschehen):

<mars\_install>/bin/mars.sh start

Das System wird gestartet.

Zum Stoppen des Systems das Skript ''<mars\_install>/bin/mars.sh stop'' ausf�hren (muss mit Rootrechten geschehen).

\subsection{Web--Komponente auf Tomcat deployen}

Tomcat muss installiert sein. Die Web-Komponente wird als \emph{.war}--Datei geliefert. Diese muss in den Tomcat--Server deployed werden.

\begin{enumerate}
\item Tomcat stoppen.
\item Falls schon mal installiert, muss im \emph{<tomcat\_install/webapps/>}-Verzeichnis die Web-Komponente erst gel�scht werden.
\item Einfach die .war--Datei in das \emph{tomcat\_install/webapps/} kopieren.
\item Den Tomcat wieder starten. Die Web--Komponente ist jetzt deployed.
\end{enumerate}

\newpage

\section{Konfiguration}\label{conf}

\subsection{Datenbank einrichten}

Um die Datenbank einzurichten, muss ein die Datenbank installiert sein und ein Hauptuser existieren. Username und Passwort m�ssen den Angaben in dem Datenbank--Konfigurationsfile ''hibernate.properties'' entsprechen.

Ist alles soweit eingerichtet muss das Datenbank--Schema importiert werden. Es liegt in ''<mars\_install>/etc/dbschema.sql''.

Mit dem Befehl ''\$ psql -d Mydatabase -f dbschema.sql'' wird das Schema in die Datenbank importiert. F�r weitere Informationen\\ siehe http://www.postgresql.org/docs/7.4/static/index.html.



\subsection{Sonstiges}

\subsubsection{Freevo einrichten}

Freevo muss installiert sein. Um die ''Mar-S'' Freevo--Oberfl�che zu integrieren:

\begin{enumerate}
\item Ins Verzeichnis \emph{python\_install/site-packages/freevo} die Files von\\ \emph{mars\_install/src/media} kopieren oder einen symbolischen Link setzen.
\end{enumerate}

\subsubsection{LIRC}

LIRC muss installiert, gestartet und f�r die benutzte Fernbedienung konfiguriert sein. N�here Informationen auf \emph{lirc.org}.

\subsection{Mailserver einrichten}

Es wird ein SMTP--Mailserver ben�tigt. Das System ''Mar-S'' muss daf�r konfiguriert werden. Hierzu existiert im Konfigurationsverzeichnis (mars\_install/etc/) des Systems die Konfigurationsdatei \emph{server\_config.properties}. Darin wird gesetzt:

\begin{enumerate}
\item mailer\_server = ''Hostname des Servers'', z.B. mail.mars.de.
\item mailer\_username = ''username''.
\item mailer\_password = ''Passwort''
\end{enumerate}

\subsection{Mediendaten verwalten}

In diesem Kapitel wird beschrieben, wie man als Benutzer seine eigenen Daten in sein pers�nliches systemeigenes Verzeichnis kopiert.

Hierzu wird Samba ben�tigt, das schon Zugriff auf die pers�nlichen Verzeichnisse hat. Diese Verzeichnisse befinden sich im Pfad unter Linux \emph{/home/username/}, wobei hier nur der eigentliche User Schreibrechte besitzt. Die Passw�rter werden vom System ''Mar-S'' gesetzt. Es k�nnen nun mittels Samba die gew�nschten Daten in die Home--Verzeichnisse der User �bertragen werden.

Die Gruppen--Verzeichnisse sind frei konfigurierbar. Sie werden im Konfigurationsverzeichnis (\emph{mars\_install/etc/}) des Systems in der Konfigurationsdatei \- \emph{server\_config.properties} mit dem Attribut \emph{fs\_groupdir = ''z.B. /usr/mars\_data''} gesetzt. Der Ordner wird automatisch angelegt.


\end{document}
