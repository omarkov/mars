\documentclass[12pt,oneside,german,a4paper]{article}
\usepackage[latin1]{inputenc}
%\usepackage{isolatin1}
\usepackage{a4}
\usepackage[T1]{fontenc}
\usepackage{babel}
\usepackage{german}
\usepackage{graphicx}
\usepackage{newcent}        % Umstellung auf Schrift Times, Wichtig f�r gute
                           % PDF Qualit�t.


\usepackage{makeidx}
\makeindex 
%----------------
% Definitionen

\setlength{\textwidth}{16cm} \setlength{\textheight}{23cm}
\setlength{\oddsidemargin}{0cm} \setlength{\topmargin}{0cm}
\setlength{\parindent}{0cm} \setlength{\parskip}{1.5ex}


\begin{document}
\onecolumn
%---------------------------------------
% Titelseite & Inhaltsverzeichnis

\title{\textbf{Administrationshandbuch}\\
      \textbf{Mar-S}}
%\author{Projektteam des Studienprojekts ''Mar-S''}
\date{\today}
% heutiges Datum: \date{\today}
\maketitle
\begin{center}
Fraunhofer Institut Office Innovation Center\\
Universit�t Stuttgart\\
Rosensteinstr. 22-24\\
70191 Stuttgart\\
\end{center}

\newpage


\tableofcontents
\newpage

\section{Komponenteninstallation}
Um eine neue Komponente zu installieren muss f�r diese in Mar-S ein Treiberprogramm geschrieben werden.
Das Treiberprogramm �bernimmt dabei drei Funktionen:
\begin{enumerate}
\item Verbindungsaufbau zum System
\item Kommunikation mit dem System
\item Verbindungabbau vom System
\end{enumerate}
\subsection{Verbindungsaufbau zum  System}
Bevor eine Komponente mit dem System kommunizieren kann, muss eine Verbindung zum System aufgebaut werden.
Dieser Verbindungsaufbau besteht aus x Schritten.
Im Code muss man zun�chst die Netzwerkklasse importieren.
\begin{verbatim}
import net.*;
\end{verbatim}
Den Verbindungsaufbau und Beschreibung der neuen Komponente kapselt man am besten in einer Funktion (im unteren Beispiel ist dies initNetwork()).
In dieser Funktion muss mann nun eine Instanz der Netzwerkkomponente holen.
\begin{verbatim}
public void initNetwork(){
NetworkFactory net = NetworkFactory.getInstance();

}
\end{verbatim}

Im n�chsten Schritt muss man eine neue Modulbeschreibung anlegen. Bei der Modulbeschreibung muss angegeben werden, ob die Komponente im Web einstellbar ist.
Hierf�r muss folgende Zeile im Code geschrieben werden
\begin{verbatim}
module.setWebModule(true);
\end{verbatim}
Die Modulbeschreibung besteht desweiteren aus Parametern und Kommandos.
Hierbei stehen folgende Parameter zu Verf�gung:
\begin{itemize}
\item Boolscher Parameter: Dieser ist zu setzen, wenn ein Komponente ein- und ausgeschaltet werden kann.
\item Numerischer Parameter: Dieser ist zu setzen, wenn eine Komponente mit einem bestimmten Zahlenwert gesteuert werden kann (Bsp. Raumtemperatur).
\item Zeichenkette: Der Komponente wird eine Zeichenkette geschickt, um sie anzusteuern. 
\item Liste: Der Komponente wird eine Liste von Zeichenketten geschickt, um sie anzusteuern (Bsp. Playlist).
\end{itemize}
Jeder dieser Parameter hat ein eigenen Konstruktor.
\begin{verbatim}
BooleanParameter(String name, Boolean default, String flags, Object obj)
NumericParameter(String name, int default, String flags, int min, int max, step int, Object obj)
StringParameter(String name, Boolean default, String flags, Object obj)
ListParameter(String name, String default, String flags, Object obj)
\end{verbatim}
Die genaueren Beschreibungen der Konstruktorparameter sind dem Entwurf Kapitel 8 zu entnehmen.

Im folgenden Codefragment wird zun�chst eine Modulbeschreibung instanziiert und dieser werden vier verschiedene Parameter zugewiesen.
\begin{verbatim}
public void initNetwork(){
        
        net = NetworkFactory.getInstance();
        
        ModuleDescription module = new ModuleDescription("TESTMODULE", "V1.0");
	module.setWebModule(true);

        BooleanParameter testBool = new BooleanParameter("testBool", true, "rw", this);
        NumericParameter testNumeric   = new NumericParameter("testNumeric", new Integer(50), "rw", 0, 255, 10, this);
        StringParameter testString = new StringParameter("testString", false, "w", this);
        ListParameter testList = new ListParamter("testList", "0", "rw", this);

        module.addInterface(testBool);
        module.addInterface(testNumeric);
        module.addInterface(testString);
        module.addInterface(testList);

}
\end{verbatim}

Neben den Parametern kann eine Modulbeschreibung auch Kommandos enthalten. Die Kommandos sind Methoden die �ber das Netzwerk aufgerufen werden k�nnen.
Das hei�t, die geschriebenen Java-Methoden m�ssen aufrufbar zur Verf�gung gestellt werden. Der Netzwerkkomponente m�ssen folgende
 Informationen �bergeben werden:
\begin{itemize}
\item Name der Methode
\item R�ckgabetyp
\item Parameterliste
\end{itemize}
Das Kommando hat folgenden Konstruktor.
\begin{verbatim}
Command(String name, Parameter returnType, Object obj)
\end{verbatim}
Dem Kommando m�ssen ebenfalls die Parameter der aufzurufenden Methode zugewiesen werden. Dies geschieht mit dem addParameter-Konstrukt.
 Hierbei wird der Name und null �bergeben. Die Parameter m�ssen in der richtigen Reihenfolge des Konstruktors hinzugef�gt werden.
Nach dem Hinzuf�gen der Kommandos kann die Komponente also gestartet werden. Der Code sieht nun so aus:

\makebox[\textwidth]{\hrulefill}
\begin{verbatim}
public void initNetwork(){
        net = NetworkFactory.getInstance();
        
        ModuleDescription module = new ModuleDescription("TESTMODULE", "V1.0");
	module.setWebModule(true);

        BooleanParameter testBool = new BooleanParameter("testBool", true, "rw", this);
        NumericParameter testNumeric   = new NumericParameter("testNumeric", new Integer(50), "rw", 0, 255, 10, this);
        StringParameter testString = new StringParameter("testString", false, "w", this);
        ListParameter testList = new ListParamter("testList", "0", "rw", this);

        module.addInterface(testBool);
        module.addInterface(testNumeric);
        module.addInterface(testString);
        module.addInterface(testList);       


	Command testCmd = new Command("testMethod", new StringParameter(), this);
	testCmd.addParameter(new NumericParameter("testp", null));


        module.addInterface(testCmd);

        net.setModuleDescription(module);
        net.startModule("localhost", 1234);
}
\end{verbatim}
Die zum Kommando geh�rende Java-Methode w�rde hier folgenderma�en aussehen:
\begin{verbatim}
public String testMethod(String testString){
   
...Hier kommt auszuf�hrende Code rein

}
\end{verbatim}
Das Modul kann sich jetzt also anmelden und kann auf Kommandos warten.
\subsection{Kommunikation mit dem System}
Bei der Kommunikation mit dem System muss man:
\begin{itemize}
\item Aufrufen von Kommandos
\item Setzen von Parametern
\end{itemize}
Hierf�r sind zwei Methoden in der NetworkFactory vorhanden.\newline
Aufrufen von Kommandos:
\begin{verbatim}
net.call(String moduleName, String methodName, new Object[]{});
\end{verbatim}
Setzen von Parametern:
\begin{verbatim}
net.set(String moduleName, String parameterName, Parameter parameterValue);
\end{verbatim}
\end{document}
